\documentclass[11pt,aspectratio=43,ignorenonframetext,t]{beamer}

% Presentation settings
\mode<presentation>{
  \usetheme[framenumber,titleframestart=1]{UoM_alex}
  \usefonttheme{professionalfonts} % using non standard fonts for beamer
  \usefonttheme{serif}
  \usepackage{fontspec}
  \setmainfont[Ligatures=TeX]{Arial}
}

% Handout settings
\mode<article>{
  \usepackage{fullpage}
  \usepackage{fontspec}
  \setmainfont[Ligatures=TeX]{Arial}
  \setlength{\parskip}{1.5\baselineskip} % correct beamer line spacings
  \setlength{\parindent}{0cm}
  \usepackage{enumitem}
  \setlist[itemize]{topsep=0pt}
}

 % Packages
\usepackage{graphicx}
\graphicspath{{./images/png}} % generic graphics path; overridden if necessary
\usepackage{amsmath}
\allowdisplaybreaks[1] % allow eqnarrays to break across pages
\usepackage{amssymb} 
\usepackage[HTML]{xcolor}
\definecolor{uomlinkblue}{HTML}{0071BC}
\usepackage{hyperref}
\hypersetup{
  colorlinks=true,
  linkcolor=uomlinkblue,
  filecolor=uomlinkblue,      
  urlcolor=uomlinkblue,
  pdflang={en-GB},
}
\usepackage[document]{ragged2e} % left aligned text for accessibility
\usepackage{tikz}
\usetikzlibrary{positioning, arrows, arrows.meta}
\usepackage{unicode-math} % unicode maths for accessibility
\usepackage{pdfcomment}   % for alt text for accessibility
\usepackage{rotating}     % allow portrait figures and tables
\usepackage{subfigure}    % allow matrices of figures
\usepackage{float}        % allows H option on floats to force here placement
\usepackage{multirow}     % allows merging of rows in tables
\usepackage{tabularx}     % allows fixed width tables
\usepackage{ctable}       % modifies \hline for use in table
\usepackage{bm}           % allow bold fonts in equations
\usepackage{pgf}          % allow graphics manipulation
\usepackage{etoolbox}
  
% Custom commands
\newcolumntype{Z}{>{\centering\arraybackslash}X}  % tabularx centered columns 

\makeatletter
  \DeclareRobustCommand{\em}
  {
    \@nomath\em
    \if b
      \expandafter\@car\f@series\@nil \normalfont
    \else
      \bfseries
    \fi
  }
\makeatother

\makeatletter
  \preto{\@verbatim}{\topsep=0pt \partopsep=0pt}
\makeatother

\def\checkmark{
  \tikz\fill[scale=0.4](0,.35) -- (.25,0) -- (1,.7) -- (.25,.15) -- cycle;
}

% Counters
\newcounter{example_number} % keep track of the example questions

% Frontmatter
\newcommand{\cmclecture}[1]{
  \title{Combinatorial Mesh Calculus (CMC): Lecture #1}
}
\author{
  Lectured by:
  \href{https://scholar.google.com/citations?user=x4R-snQAAAAJ&hl=en}
  {Dr. Kiprian Berbatov}$^1$\\
  \smallskip
  Lecture Notes Compiled by:
  \href{https://scholar.google.com/citations?user=CoIpITkAAAAJ&hl=en}
  {Muhammad Azeem}$^1$\\
  \smallskip
  Under the supervision of:
  \href{https://scholar.google.co.uk/citations?user=3nWJe5wAAAAJ&hl=en}
  {Prof. Andrey P. Jivkov}$^1$\\
  \smallskip
  {\tiny $^1$Department of Mechanical and Aerospace Engineering,
    The University of Manchester, Oxford Road, Manchester M13 9PL, UK}
}

% Special frames
\newcommand{\cmctitleframe}{
  \titlepage
  \begin{tikzpicture}[remember picture,overlay]
    \node[anchor=south east] at (current page.south east) {
      \href{https://youtube.com/@kipi.berbatov}{
        \includegraphics[width=1.5cm]{youtube-icon.png}
      }
    };
  \end{tikzpicture}
}
\newcommand{\cmcendframe}{
  \begin{figure}
    \centering
    \includegraphics[width=0.85\linewidth]{Thanks.png}
  \end{figure}
}
